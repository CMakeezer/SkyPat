\documentclass[final]{ols}
\usepackage{color,framed,url,zrl}
\usepackage[colorlinks=true]{hyperref}
\usepackage{listings}
\definecolor{shadecolor}{rgb}{0.9,0.9,0.9}
\ifpdf\usepackage[pdftex]{graphicx}\else\usepackage{graphicx}\fi

\begin{document}

\title{SkyPat: C++ Performance Analysis and Testing Framework}
\subtitle{}

\author{
	Ping-Hao Chang \\
	{\em Skymizer}\\
	{\tt\small peter@skymizer.com}\\
\and
	Kuan-Hung Kuo\\
	{\em Skymizer}\\
	{\tt\small ggm@skymizer.com}\\
\and
	Der-Yu Tsai\\
	{\em Skymizer}\\
	{\tt\small a127a127@skymizer.com}\\
\and
	Kevin Chen\\
	{\em Skymizer}\\
	{\tt\small kevin@skymizer.com}
\and
	Luba Tang\\
	{\em Skymizer}\\
	{\tt\small luba@skymizer.com}\\
}
\shortauthor{P.H.\ Chang \& K.H.\ Kuo \& D.Y.\ Tsai \& K.\ Chen \& Luba W.L.\ Tang}

\date{} % Do not print the date

\maketitle

%\thispagestyle{empty} % Do not use \thispagestyle in your paper.

\begin{abstract}
This paper introduces SkyPat, a C++ performance analysis toolkit on Linux.
SkyPat combines unit tests and perf\_event to give programmers the power of white-box performance analysis.

Unlike tranditional tools that manipulate entire program as a black-box, SkyPat works on a region of code like a white-box.
It is used as a normal unit test library.
It provides macros and assertions, which perf\_events are embeded in, to ensure correctness and to evaluate performance of a region of code.
With perf\_events, SkyPat can evaluate running time precisely without interference to scheduler.
Moreover, perf\_event also gives SkyPat accurate cycle counts that are useful for tools who are sensitive to variance of timing, such as compilers.

We develop SkyPat under the new BSD license, and it is also the unit-test library of the "bold" project.
\end{abstract}

\section{Introduction}
SkyPat is developed by a group of the compiler developers to satisfy compiler developers' needs.
From compiler developers' view, correctness and performance evaluation are grand challenges for engineering.
Compiler optimizations have no guarantee of performance improvement.
Sometimes optimizations degrade performance, and sometimes they introduce new faults in a program.
Compiler developers need a tool to verify correctness and performance of each compiler optimizations at the same time.

Traditional tools that evaluation the whole program don't fit our demands.
Compilers perform optimization based on knowledge it has to a region of code, such as loops or data flows.
We needs libraries that can evaluate only a region of code that optimization is interesting in.

For compiler developers, integrating unit-test and performance evaluation libraries for a piece of code is very rational.
SkyPat is such library: by using SkyPat, users can evaluate correctness and performance by writing test-cases to evaluate a region of code.

Usually, unit-test tools and performance evaluation tools are separated tools.
For example, \textit{GoogleTest} \cite{Google-test} is well-known C++ unit-test framework.
\textit{GoogleTest} can evaluate correctness but cannot evaluate performance.
Besides, \textit{perf} \cite{perf-tools} is well-known performance evaluation toolkit in Linux.
\textit{perf} can evaluate performance of programs, including its running time, cycles and so on.
Although \textit{perf} can evaluate whole program, using \textit{perf} to evaluate a region of code is difficult.

In this paper, we introduces SkyPat, which combines unit-test and performance evaluation.
Programmer only need to write and execute unit-tests and they can get correctness and performance.
With the help of perf\_event of Linux kernel, SkyPat can provide precise timer and additional runtime information to measure a region of code.
By integrating unit-test and performance evaluation, SkyPat make evaluation of a region of code easier.

The organization of this paper is as follows.
Related work is in Section 2.
We presents SkyPat's design and implementation in Section 3 and shows the evaluation results in Section 4.
At last, we conclude this paper in Section 5.

\section{Related work}

Oprofile\cite{oprofile} and \textit{perf} are two most popular performance evaluation tools.
Oprofile is capable to evaluate not only the whole system but also a single program.
Before versions 0.9.7 and earlier, it was based on sampling-based daemon that collects runtime information.
Because sampling-based daemon wasted system resources, Linux community creates new interfaces, called ``Performance Counter''\cite{performance-counter-linux} or ``perf\_event''.
After that, Oprofile is ``perf\_event''-based in the later version.

By the success in ``Performance Counter'', Linux community builds another performance evaluation tool, called \textit{perf}, based on ``Performance Counter'', too.
\textit{Perf} is a performance evaluation toolkit with no daemon to collect runtime information.
\textit{perf} gets runtime information by kernel directly rather than collecting by daemon, therefore, \textit{perf} eliminates lots of overhead to bookkeep profiling information and becomes faster.

Both Oprofile and \textit{perf} evaluate performance of the entire program, not a region of code.
And of course, it makes some efforts to integrate them with uni-test frameworks.

Regarding unit-test frameworks, \textit{GoogleTest} becomes popular and adapted by many project recently.
\textit{GoogleTest} is a xUnit test framework for C++ program.
By providing ASSERT and EXPECT macros, \textit{GoogleTest} helps programmers to verify program's correctness by writing test-cases.
While executing test-cases, a program stops immediately if it meets a fatal error.
If program faces non-fatal error, program will show the runtime value and expected value on the screen to programmer.

\section{Implementation}

SkyPat is a C++ performance analyzing and testing framework on Linux platforms.
We refer the concept of Google Test and extend its scope from testing framework into Performance Analysis Framework.
With the help of SkyPat, programmers who wants to analyze their program, only need to write test cases via SkyPat, and SkyPat can analyze programs' correctness and performance.
That is to say, SkyPat integrates unit-test framework and performance evaluation tool together.

SkyPat provides several macro to integrate test cases and main program, including ASSERT/EXPECT for correctness and PERFROM for performance.
ASSERT and EXPECT are assertions for condition testing and PERFORM wraps a block of code for performance testing.
ASSERT is assertion for fatal condition testing and EXPECT is non-fatal assertion.
That is to say, if the condition of ASSERT fails, the test fails and stops immediately.

When the condition of EXPECT fails, it displays on screen to indicate that is a non-fatal failure.
The whole test keeps running and is not considered as a failure.
PERFORM evaluates performance rather than correctness.
With the help of PERFORM, user can measure the performance of code within a test.

To get the performance of the region of code, SkyPat uses macro to assemble ``perf\_event'' interface into the test-case.
SkyPat uses ``perf\_event'' interface to register events to be monitored when program initials.
When program executes at the beginning of the region of code to be monitored, SkyPat will send a system call to kernel to enable the counter to gather process's runtime information, such as execution time.
When program executes in the end of the region of code, SkyPat will send a system call to kernel to disable the counter.
SkyPat gets the difference of time between begin and end to calculate the value of runtime information during executing the region of code.

Because ASSERT/EXCEPT/PERFORM are macros, when user writing their test-case, they only have to let the region of code placed between braces of the macro.
When user compiles their test-case, compiler will integrate all SkyPat framework to the test-case.

\section{Unit-test framework}
For those who wants to use SkyPat as unit-test and performance evaluation framework, there are three basic steps to use SkyPat:
\begin{itemize}
\item Writing test cases including ASSERT, EXPECT, and PERFORM macros
\item Compiling these cases with your main program and the shared library of pat.
\item Execute the binary file and you can get its correctness and performance.
\end{itemize}
This section shows how to use SkyPat to evaluate another program's correctness and performance.

\subsection{Declare test-case and test function}

SkyPat uses test-case to evaluate testee's correctness and performance.
Test-case is grouped by a bunch of "test functions".

In Figure~\ref{aircraftcase}, it shows how to define test-case and test-function.

\begin{figure}[h]
\lstset{language=C++}
\begin{lstlisting}[frame=single]
PAT_F(AircraftCase, take_off_test)
{
   // Test Code
}

PAT_F(AircraftCase, landing_test)
{
   // Test Code
}
\end{lstlisting}
\caption{Example for declaring a test}
\label{aircraftcase}
\end{figure}

PAT\_F macro declare a test, with two parameters: test-case's name and test-function's name.
In Figure~\ref{aircraftcase}, ``AircraftCase'' is the name of test-case.
``take\_off\_test'' and ``landing\_test'' are the names of test-function belongs to a test-case, ``AircraftCase''.
User of SkyPat should put logically related tests into the same test-case.
In test-function, user can evaluate correctness and reliability by inserting macros.
Guide for these macros is described in following section.

\subsection{Correctness evaluation}
We accept the concept from \textit{GoogleTest} for our correctness evaluation.
For correctness evaluation, user can assert fatal or non-fatal condition in the region of code in test-function.

The macro for fatal condition is ASSERT.
Programmer tests a class or a function by making ASSERT assertions about its behavior.
There are two kinds of ASSERT macros, ASSERT\_TRUE and ASSERT\_FALSE.
If the condition of ASSERT fails, the test will stop and exit the test immediately.
The macro for non-fatal condition is EXPECT.
There are two kinds of EXPECT macros, EXPECT\_TRUE and EXPECT\_FALSE.
If the condition of EXPECT fails, the test will not stop but keep execute and display the expected result and real result on screen.

\begin{figure}[h]
\lstset{language=C++}
\begin{lstlisting}[frame=single]
PAT_F(SeriesCase, fibonacci_test)
{
  ASSERT_TRUE(0 != fibonacci(10));
  EXPECT_TRUE(2 == fibonacci(10));
  ASSERT_TRUE(3 == fibonacci(10));
}
\end{lstlisting}
\caption{Example of assertions}
\label{assert_example}
\end{figure}

In Figure~\ref{assert_example}, it shows how fatal and non-fatal assertions work.
Function ``fibonacci'' is the testee function which can be replaced any function of testee program by user.
There are three assertion, including fatal and non-fatal, and the condition of the second and third condition are fail.

\begin{figure}[h]
\lstset{language=sh}
\begin{lstlisting}[frame=single]
[ RUN      ] MyCase.fibonacci_test
[  FAILED  ]
main.cpp:53: error: failed to expect
Value of: 2 == fibonacci(10)
  Actual:   false
  Expected: true
main.cpp:54: fatal: failed to assert
Value of: 3 == fibonacci(10)
  Actual:   false
  Expected: true
\end{lstlisting}
\caption{Output of Figure~\ref{assert_example}}
\label{assert_example_output}
\end{figure}

In Figure~\ref{assert_example_output}, it shows the output of Figure~\ref{assert_example}.
When program executes EXPECT assertion, although its condition is fail, SkyPat displays its status as ``error: failed to expect'' and keep s executing following code.
However, when program executes the last ASSERT assertion whose condtion is fail, because it is fatal assertion, the program stops immediately and displays its status as ``fatal: failed to assert''.
With the help of these two assertions, user can insert them into their test-cases to verify correctness of their program.

\subsection{Performance evaluation}
For performance evaluation, user can insert PERFROM macro into the region of code where user wants to evaluate its performance.

PERFORM macro measure the performance of a block of code wrapped by it.
SkyPat will evaluate its runtime behavior between two braces of PERFROM.

\begin{figure}[h]
\lstset{language=C++}
\begin{lstlisting}[frame=single]
PAT_F(MyCase, fibonacci_perf_test)
{
  PERFORM {
    fibonacci(40);
  }
}
\end{lstlisting}
\caption{Example of PERFORM}
\label{perform_example}
\end{figure}

In Figure~\ref{perform_example}, it shows how to use PERFORM macro to write a test-function.
It uses PERFORM to wrap the region of code, which is ``fibonacci(40)'' in this case.
And SkyPat will use ``perf\_event'' to measure the runtime behavior during executing the region of code.

\begin{figure}[h]
\lstset{language=sh}
\begin{lstlisting}[frame=single]
[ RUN      ] MyCase.fibonacci_test
[CXT SWITCH]       3
[ TIME (ns)]  2363214415
\end{lstlisting}
\caption{Output of Figure~\ref{perform_example}}
\label{perform_example_output}
\end{figure}

Figure~\ref{perform_example_output} shows the result of Figure~\ref{perform_example}.
It measures the execution time and the number of context-switch during the region of code.
User does not need to control complicate \textit{perf} program to monitor the region of code or even use ``perf\_event'' system by themselves.
User only needs to use macro, just likes what they custom in unit-test framework, and they can easily get the runtime information of the region of code.

\subsection{Integrate all test-cases}

\begin{figure}[h]
\lstset{language=C++}
\begin{lstlisting}[frame=single]
int main(int argc, char* argv[])
{
  pat::Test::Initialize(&argc, argv);
  pat::Test::RunAll();
}
\end{lstlisting}
\caption{Example of Initialize and RunAll}
\label{main_example}
\end{figure}

To integrate all test-case, user should call ``Initialize'' and ``RunAll'' in their main program.
Otherwise, user doesn't need to concern about register their test-case into SkyPat.

``Initialize(\&argc, argv)'' helps pat to initial the output interface of pat. If you do not call Initialize, you can not see the output of pat.
And the ``RunAll()'' function magically knows about all the tests you defined.
``RunAll()'' runs all the tests you've declared, prints the results and returns zero only if all tests are succeed. Otherwise, it returns one.

\section{Conclusion and Future Works}
By integrating unit-test framework and performance evaluation tool, user can get correctness and performance number of the region of code by writing test-cases.
Users can get the performance between region of code defined by themselves.
For programs which needs high precise timing information and other runtime information of the region of cdoe, such as compiler, SkyPat can give them more ability to measure the bottleneck of regions of a program.

\begin{thebibliography}{99}  % The section of reference
\addcontentsline{toc}{section}{Reference}  % Add ``Reference'' into the table of contents
\bibitem{perf-tools}
Arnaldo Carvalho de Melo, Redhat, ``The New Linux `perf' tools'' in \emph{17 International Linux System Technology Conference (Linux Kongress), 2010}

\bibitem{Google-test}
GoogleTest, Google, \texttt{\small https://code.google.com/p/googletest/}

\bibitem{oprofile}
Oprofile, \texttt{\small \newline http://oprofile.sourceforge.net/about/}

\bibitem{performance-counter-linux}
Ingo Molnar, ``Performance Counters for Linux'' in \emph{Linux Weekly News, 2009}, \texttt{\small http://lwn.net/Articles/337493/}

\end{thebibliography}

\end{document}
